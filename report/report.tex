\documentclass[a4paper,12pt]{report}

\input{header.tex}


\title{Рассчётно-пояснительная записка по курсовой работе "Разработка почтового сервера"}
\author{Кисленко Максим Германович}

\begin{document}

\maketitle

\tableofcontents

\cleardoublepage
\phantomsection
\addcontentsline{toc}{chapter}{Введение}
\chapter*{Введение}

% Два предложения о содержании отчёта. Для нового абзаца в исходном тексте должна быть пустая строка.

% Это~-- шаблон отчёта (вот как оформляется длинное тире, перед котрым идёт неразрывный пробел).

Курсовая работа предполагает разработку собственного почтового сервера, который будет использоваться для пересылки электронных писем (email-ов) в IP-сети. Вся работа поделена на две приблизительно одинаковые части - разработка почтового сервера, осуществляющего приём писем, и разработка почтового клиента, который переотправит письма другим почтовым серверам в зависимости от получателей-адресатов. Эти части разрабатываются независимо друг от друга двумя студентами, а потом интегрируются в единую систему. Команде студентов дополнительно присваивается вариант, который определяет основные требования к реализации или функциональности.

При реализации подразумевается, что используются низкоуровневые системные вызовы и, что сервер работает под управлением POSIX-совместимой операционной системы. Последнее условие необходимо, чтобы обеспечить переносимость созданного ПО.

В данном документе будет рассмотрена разработка почтового сервера, с условиями от варианта 12. Условия включают в себя следующие пункты:
\begin{itemize}
	\item сервер должен обрабатывать входящие соединения в единственном рабочем потоке с применением вызова poll. Такое условие подразумевает использование сокетов в режиме неблокирующего ввода-вывода, так как иначе, единственный поток бы блокировался на одном клиенте и не мог бы считаться полноценным сервером;
	\item логирование событий и ошибок в работе сервера должно происходить в отдельном процессе. Такое техническое решение увеличивает сложность реализации, но и делает сервер более производительным (отсутствуют операции ввода-вывода на жесткий диск);
	\item проверка обратной зоны dns не предполагается.
\end{itemize}
Независимо от условий варианта, при постановке задачи заданы требования к выполнению данной работы. А именно:
\begin{itemize}
	\item в качестве протокола пересылки электронных сообщений предписано использовать протокол SMTP (simple mail transfer protocol). Вся необходимая информация по нему доступна в RFC 5321;
	\item для локального хранения писем использовать механизм Maildir;
	\item конфигурация сервера должна быть вынесена в отдельный конфигурационный файл;
	\item необходимо разработать автоматическое тестирование созданного ПО (системные, unit-тесты);
	\item необходимо автоматизировать сборку проекта из исходников.
\end{itemize}

В поставленной задаче довольно жестко регламентированы используемые технологии. В качестве основного языка программирования - язык Си стандарта 99-го года. Для автоматизации тестирования - cunit или скриптовые языки (python, ruby или другое). Для автоматизации сборки - только GNU Make. Также готовое ПО необходимо протестировать на наличие утечек памяти с помощью утилиты valgrind.

Готовая программа должна обрабатывать входящие соединения, обслуживать их в соответствии с правилами сессии SMTP, сохранять письма локально в Maildir вместе с необходимыми заголовками, и передавать письма на дальнейшую отправку SMTP-клиенту. Далее в этом документе детально рассмотрены:

\begin{itemize}
	\item Пользовательское взаимодействие с сервером, его достоинства, недостатки и существующие аналоги в аналитическом разделе;
	\item Проектирование и принятые программные решения при создании ПО в конструкторском разделе;
	\item Используемые инструменты и технологии, порядок эксплуатации созданного ПО в технологическом разделе;
\end{itemize}

% А вот так оформляются списки: \begin{itemize} \item элемент списка; \item последний элемент списка. \end{itemize}

% Нумерованный список выглядит следующим образом. \begin{enumerate} \item Первый элемент. \item Второй элемент. \end{enumerate}


\chapter{Аналитический раздел}

\section{Предметная область}

В результате проведённого исследования были выделены следующие сущности предметной области (рисунок ~\ref{fig:entities}). Здесь и далее все диаграммы выполняются в нотациях UML, если иное не указано явно.

\begin{figure}
\centering
\includegraphics[width=\textwidth]{diagramms/entities.png}
\caption{Основные сущности предметной области}
\label{fig:entities}
\end{figure}

\section{Достоинства и недостатки реализуемого сервера}

Достоинства реализованного ПО проистекают из его архитектурных решений:
\begin{enumerate}
    \item благодаря однопоточной реализации, в сервере нет потерь времени на переключение контекстов, ожидание блокировок и проблем с разделяемой памятью;
    \item благодаря неблокирующему вводу-выводу, сервер может с очень высокой производительностью обслуживать запросы клиентов, при условии, что сама обработка занимает мало времени. Последнее обеспечивается архитектурой протокола SMTP, который изначально построен по принципу: "многих быстрых этапов";
    \item благодаря логированию в отдельном процессе, сервер не тормозится на IO операциях с жестким диском. Основной процесс просто отправляет буфер в очередь и продолжает исполнение, а записью в файл занимается отдельный процесс. Такое решение хорошо и тем, что уменьшает связность и зависимость программных компонент.
\end{enumerate}

Как и у любой системы, наряду с достоинствами есть и недостатки.
\begin{enumerate}
    \item из-за однопоточности сервера затруднено решение вопросов с отказоустойчивостью, так как при появлении фатальной ошибки в основном процессе, весь сервер "падает" (с потерей всех данных о текущих клиентах) и перестаёт обслуживать клиентов;
    \item также затруднено обеспечение масштабируемости при росте нагрузки. Единственное решение - поднимать несколько демонов-серверов на разных портах и производить балансировку, что очень накладно по расходуемым ресурсам.
\end{enumerate}

Данные недостатки уменьшены в существующих решениях, к примеру в сервере Nginx, благодаря более сложной архитектуре. Она включает не только мастер-процесс для IO-операций на неблокирующих сокетах, но и процессы worker-ы.


\section{Программы-аналоги}

Схожим функционалом обладают две широко-используемые программы: exim и postfix.


\chapter{Конструкторский раздел}

\section{Архитектура сервера}

Реализация почтового сервера разделена на несколько функциональных модулей с собственными заголовочными файлами. В server.h и server.c выполняется установка соединений с клиентами, получение от них текстовых команд и вызов SMTP-парсера для формирования ответа. Он также хранит для каждого из соединений запись о текущем состоянии (это связано с тем, что SMTP является stateful-протоколом). Сервер является однопроцессным, однопоточным и работает с сокетами клиентов в режиме неблокирующего ввода-вывода. Для получения событий о готовности ввода-вывода применяется системный вызов \textit{poll()}. Подобная архитектура применяется в http-серверах nginx (с процессами-воркерами для обработки запросов) и node.js.

Опишем алгоритм работы сервера с помощью python-подобного псевдокода. Пусть \textit{fd} будет основным сокетом, на котором сервер сделал вызовы \textit{bind()} и \textit{listen()}, тогда:

\begin{verbatim}
sockets = []
sockets.append({fd: fd, events: POLLIN})
active_cnt = 1

while True:
    code = poll(sockets, active_cnt, timeout)
    if code == 0:
        # timeout expired
        return 0

    quit = []
    for sock in sockets:
        if sock == fd:
            accept_new_conns(fd)
        else:
            state = get_state(sock)
            code = handle_known_conn(state) 
            if code < 0:
                # received QUIT command
                quit.append(sock)

    if quit:
        for sock in quit:
            sockets.remove(sock)
\end{verbatim}

Здесь и далее в листингах не показываются обработки ошибок, только основные действия. Чтобы принять новые соединения сервер выполняет внутри функции \textit{accept_new_conns} системный вызов \textit{accept}.

\begin{verbatim}
while True:
    new_sock = accept(fd)
    if new_sock < 0:
        return new_sock
    else:
        create_state(new_sock)
        send(new_sock, 'Welcome!')

        sockets.append(new_sock)
        active_cnt += 1
\end{verbatim}

Если поступают данные от клиента, то сервер вызовет \textit{handle_new_conn}, предварительно восстановив состояние этого клиента по сокету. Для получения данных используется вызов \textit{recv}. Для анализа полученных от клиента данных функция SMTP-парсера \textit{handle_request}.

\begin{verbatim}
input = ''

while True:
    chunk = recv(sock, input, 1000)
    if chunk:
        input += chunk
    else:
        break

if input:
    resp = handle_request(input)
    send(sock, resp)

\end{verbatim}

% Рис.~\ref{fig:fsm} нагенерил самодельный \textit{fsm2dot} из \textit{autogen} и \textit{dot2tex} на пару \textit{dot}. Никто не мешает изменить параметры типа \textit{rankdir} прямо в \textit{fsm2dot}, если он будет лучше смотреться, например, сверху-вниз.

% \begin{figure} \centering \includegraphics[width=\textwidth]{include/server_def_dot.pdf} \caption{Состояния сервера} \label{fig:fsm} \end{figure}

\section{Обработка команд протокола}

Далее SMTP-парсер, анализирует полученную строку и в зависимости от неё и текущего состояния конкретного клиента выбирает выполняемое действие, новое состояние и нужный ответ клиенту. Парсер реализован в отдельном модуле из файлов smtp.h и smtp.c. По сути он представляет собой конечный автомат и может быть описан диаграммой состояний переходов (рисунок ~\ref{fig:smtp_fsm}).

\begin{figure}
\centering
\includegraphics[width=\textwidth]{diagramms/smtp_fsm.png}
\caption{Конечный автомат протокола SMTP}
\label{fig:smtp_fsm}
\end{figure}

На диаграмме красным цветом выделен путь по графу, который соответствует обычной отправке письма с одним получателем. Синим - другие возможные переходы. На диаграмме также не показано несколько команд, которые не меняют состояния - это QUIT (команда серверу о закрытии соединения), NOOP (команда ничего не делать - no operation) и VRFY (доступна для вызова в состояниях READY и NEED_SENDER).

Некоторые из команд используют параметры. Рассмотрим их подробнее:
\begin{itemize}
	\item \textit{HELO domain} и \textit{EHLO domain}, где domain определяется с помощью регулярного выражения \textit{([a-zA-Z0-9]([a-zA-Z0-9\-]{0,61}[a-zA-Z0-9])?\.)+[a-zA-Z]{2,6})}. Команды инициируют smtp-сессию. В ответ на EHLO ожидается развернутый ответ со списком поддерживаемых расширений;
	\item \textit{MAIL FROM: <email>} и \textit{RCPT TO: <email>}, где email определяется с помощью регулярного выражения \textit{([-A-z0-9.]+@([A-z0-9][-A-z0-9]+\.)+[A-z]{2,4})}. Команды задают отправителя и получателя. Получателей может быть несколько.
	\item \textit{VRFY email_part}, где email_part определяется по \textit{([A-z0-9.@-_]+)}. Команда возвращает информацию по известным для сервера пользователям в формате <полное имя> <email>, в email-е которых встретилась передаваемая параметром последовательность символов.
\end{itemize}

На команды клиента сервер отвечает предопределенными в спецификации SMTP кодами. Далее приведем список поддерживаемых команд.
\begin{itemize}
	\item 221 Closing transmission channel - закрытие соединения по инициативе сервера;
	\item 250 OK - команда корректна и принята к исполнению;
	\item 354 Start mail input; end with <CRLF>.<CRLF> - ответ при переходе в состояние GET_DATA;
	\item 451 Requested action aborted: error in processing - ошибки при выполнении дисковых операций с maildir;
	\item 455 Server unable to accommodate parameters - возвращается в случае, когда получателей больше максимума;
	\item 500 Invalid command - неизвестная команда SMTP;
	\item 501 Invalid argument - отсутствие или некорректный аргумент в команде;
	\item 503 Bad sequence of commands - неожиданная команда для текущего состояния;
	\item 550 Sender unknown - email отправителя не является известным для smtp-сервера (известные записаны в файле usersinfo.txt);
	\item 552 Requested mail action aborted: exceeded storage allocation - ошибка выделения памяти, например для тела письма;
\end{itemize}

% \begin{description} \item[Команда выхода из сеанса] \input{include/re_cmd_quit_re.tex} \item[Команда передачи имени пользователя] \input{include/re_cmd_user_re.tex} \end{description}

% Для грамматики можно использовать вставку из файла и оформление \textbackslash{}begin\{verbatim\} и \textbackslash{}end\{verbatim\} или пакет \textit{listings}\footnote{На дворе XXI век, но пакет \textit{listings} всё ещё не пашет с русскими комментариями без бубна, и лично я его пока не победил.}.

% Для примера воспользуемся автоматической вставкой файла описания параметров программы (не забудьте перенести это в технологический раздел) через утилитку \textit{src2tex}.

% \input{include/checkoptn.def.tex} \lstset{language=C} \lstinputlisting{../src/checkoptn.def}

\section{Работа с Maildir}

Maildir - распространенный формат хранения электронной почты, который обеспечивает целостность передаваемых писем и самих почтовых ящиков за счет того, что операции по блокировке и перемещению отданы локальной файловой системе. Каждое письмо в maildir представлено отдельным файлом с уникальным именем (в работе используется UUID), который при получении записывается в подкаталог tmp, затем по окончанию записи помещается в new, где его найдет почтовый клиент. Схема Maildir представлена на рисунке ~\ref{fig:maildir}.

\begin{figure}
\centering
\includegraphics[width=\textwidth]{diagramms/maildir.png}
\caption{Организация maildir}
\label{fig:maildir}
\end{figure}

Отдельный каталог Maildir создаётся для каждого получателя. Если получателей у одного письма несколько, то запишется несколько файлов с одинаковым содержимым для разных пользователей. Формат файла включает помимо тела письма ещё и SMTP-заголовки, а также дополнительные поля Subject (берется первая строка тела) и Date (дата получения письма сервером).


\section{Логирование в отдельном процессе}

Процесс, в котором происходит логирование, стартует из основного по вызову \textit{fork()}. Соответственно, он является дочерним и разделяет с родителем обработчики прерываний (к примеру, SIGINT, послылаемый при нажатии Ctrl-C в терминале) и глобальную переменную-указатель на прочитанную конфигурацию.

Взаимодействие между процессами организовано с помощью System V IPC. Этот механизм более древный, чем Posix MQ и более распространен среди операционных систем (к примеру, posix mq не поддерживается OSX и MacOS). Реализация логирования для сервера расположена в модуле с файлами logger.h и logger.c.

\begin{verbatim}
#include <sys/ipc.h>
#include <sys/msg.h>

key_t key = ftok("/tmp", 'S');
int msg_queue = msgget(key, 0644 | IPC_CREAT);
FILE *log_file = fopen(log_file_name, "w");

char buffer[512];
while (1) {
    int res_code = msgrcv(msg_queue, &buffer, sizeof(buffer), 0, 0);
    if (strcmp(buffer, "Stop") == 0) {
        break;
    }

    char now[40];
    formatted_now(now, 40);
    fprintf(log_file, "[%s] %s\n", now, buffer);
    fflush(log_file);
}
\end{verbatim}

Для того чтобы отправить сообщение в лог, основной процесс должен сначала подключиться к очереди.
\begin{verbatim}
key_t key = ftok("/tmp", 'S');
int msg_queue = msgget(key, 0644 | IPC_CREAT);
int res_code = msgsnd(msg_queue, msg, strlen(msg), 0);
\end{verbatim}

Процесс логирования останавливается по инициативе основного при посылке строки "Stop". Это используется, если сервер завершил выполнение по таймауту.


\chapter{Технологический раздел}

% Нужно отметьть, что символ <<\_>> необходимо оформлять как <<\textbackslash\_>>.

\section{Используемые технологии и аппаратура}

Разработанное ПО разрабатывалось и тестировалось на компьютере Macbook PRO модели 2015 года с процессором Intel Core i5, 8 GB оперативной памяти и 128 GB SSD, под управлением операционной системы MacOS High Sierra 10.13.

Для того чтобы собрать исполнемый файл из исходных текстов необходимо наличие в системе GNU make (>= 3.81) и компилятора языка си (>= llvm 9.0, если установлен clang и >= 4.2.1, если используется gcc). В проекте также приложена конфигурация gmake (файл smtp_server.pro), который использовался для быстрого запуска при разработке. Запущенный сервер пишет в консоль идентификатор собственного процесса, процесса логирования и строку "Server started". Подробнее про сборку в следующей секции.

Для чтения файла конфигурации сервер использует библиотеку libconfig версии >= 1.7.2, которая также должна быть установлена в системе для корректного запуска. Путь к заголовочным файлам библиотеки нобходимо явно указать в Makefile в переменной \textit{INCPATH}. Значение по умолчанию - /usr/local/Cellar.

Чтобы убедиться в корректной работе собранного сервера, можно запустить системные тесты. Они написаны на языке python и требуют версию интерпретатора >= 3.4. Для запуска достаточно просто выполнить \textit{python3 as_user.py host port}, заменив параметры командной строки на те хост и порт, на которых запущен сервер. Подробнее про тестирование ниже.

\section{Сборка программы}

Сборка осуществляется с помощью утилиты make. Конфигурационный файл с описанием целей приложен в каталоге проекта /build. Для его исполнения достаточно команды \textit{make all}.

Помимо создания исполняемого файла, Makefile прикладывает файлы конфигурации. Для очистки предусмотрена цель \textit{make clean}.

\section{Конфигурация в отдельном файле}

Пример файла конфигурации для почтового сервера в conf/settings.cfg. Содержимое поделено на три секции - настройки для собственно сервера, smtp-парсера и для процесса логирования.

\begin{verbatim}
version = "1.0";

server = {
    port = 9091;
    timeout_sec = 200;
    max_clients = 100;
};

smtp = {
    name = "smtp.maxim.ru";
    maildir = "/Users/maksimkislenko/smtp_env/mails";
    userinfo = "/Users/maksimkislenko/smtp_env/smtp_server/conf/userinfo.txt";
};

logger = {
    path = "/Users/maksimkislenko/smtp_env/log.txt";
};

\end{verbatim}

Чтение и разбор файла конфигурации осуществляет библиотека libconfig при запуске основного процесса. По умолчанию ищется файл settings.cfg в текущей директории. Однако можно поменять это поведение передав параметром командной строки путь к собственным настройкам. Также есть возможность выводить доп. информацию в терминал - для этого при запуске достаточно указать \textit{--verbose}.

\section{Автоматические тесты}

Тесты сгруппированы в сценарии. Каждый состоит из одного соединения и нескольких посылаемых команд. Полученные ответы сравниваются с ожидаемыми (ищутся вхождения ключевых слов и необходимый smtp-код). Если хоть одно из вхождений не найдено, будет сгенерировано исключение с информативным сообщением.

Код сценариев приведен в следующем листинге:

\begin{verbatim}
def script_1():
    print '=== Script 1 started (Test connection - HELO, QUIT, then write) ==='

    with SMTP_Client(HOST, PORT) as client:
        test_smtp_output('hello', client.hello(), ['250', 'ok'])
        test_smtp_output('quit', client.quit(), ['221', 'closing'])
        try:
            client.hello()
        except Exception as e:
            test_smtp_output('after close', str(e), ['errno 54', 'connection reset'])
    print ''

def script_2():
    print '=== Script 2 started (Test EHLO, RSET, NOOP commands) ==='

    with SMTP_Client(HOST, PORT) as client:
        test_smtp_output('1st ehlo', client.hello_extended(), ['250-', 'pipelining', '8bitmime', 'vrfy'])
        test_smtp_output('1st noop', client.no_operation(), ['250', 'ok'])
        test_smtp_output('rset', client.reset(), ['250', 'ok'])
        test_smtp_output('hello', client.hello(), ['250', 'ok'])
        test_smtp_output('2nd noop', client.no_operation(), ['250', 'ok'])
        test_smtp_output('2nd ehlo', client.hello_extended(), ['250-', 'pipelining', '8bitmime', 'vrfy'])
        test_smtp_output('quit', client.quit(), ['221', 'closing'])
    print ''

def script_3():
    print '=== Script 3 started === (Test ordinal mailing)'

    with SMTP_Client(HOST, PORT) as client:
        test_smtp_output('hello', client.hello(), ['250', 'ok'])
        test_smtp_output('sender', client.mail_from(), ['250', 'ok'])
        test_smtp_output('receiver', client.recipient_to(), ['250', 'ok'])
        test_smtp_output('data', client.data(), ['250', 'ok'])
        test_smtp_output('quit', client.quit(), ['221', 'closing'])
    print ''

def script_4():
    print '=== Script 4 started === (Test many recipients)'

    with SMTP_Client(HOST, PORT) as client:
        test_smtp_output('hello', client.hello(), ['250', 'ok'])
        test_smtp_output('sender', client.mail_from(), ['250', 'ok'])
        
        test_smtp_output('receiver', client.recipient_to(email='<test1@mail.ru>'), ['250', 'ok'])
        test_smtp_output('receiver', client.recipient_to(email='<test2@mail.ru>'), ['250', 'ok'])
        test_smtp_output('receiver', client.recipient_to(email='<test3@mail.ru>'), ['250', 'ok'])
        
        test_smtp_output('data', client.data(), ['250', 'ok'])
        test_smtp_output('quit', client.quit(), ['221', 'closing'])
    print ''

def script_5():
    print '=== Script 5 started === (Errors 5xx)'

    with SMTP_Client(HOST, PORT) as client:
        test_smtp_output('mail_from before hello', client.mail_from(), ['503', 'bad', 'sequence'])
        test_smtp_output('invalid hello', client.hello(cmd='HELLO'), ['500', 'invalid', 'command'])
        test_smtp_output('hello', client.hello(), ['250', 'ok'])

        test_smtp_output('invalid sender argument', client.mail_from(email='test'), ['501', 'invalid', 'argument'])
        test_smtp_output('unknown sender', client.mail_from(email='<test@mail.ru>'), ['550', 'unknown', 'sender'])
        test_smtp_output('sender', client.mail_from(), ['250', 'ok'])
        test_smtp_output('hello after mail_from', client.hello(), ['503', 'bad', 'sequence'])
        
        test_smtp_output('no receiver', client.recipient_to(email=''), ['501', 'invalid', 'argument'])
        test_smtp_output('quit', client.quit(), ['221', 'closing'])
    print ''

def script_6():
    print '=== Script 6 started === (Verify command)'

    with SMTP_Client(HOST, PORT) as client:
        test_smtp_output('hello', client.hello(), ['250', 'ok'])
        test_smtp_output('verify known', client.verify(), ['250', 'chesalin', 'denis', '<chesalka@mail.ru>'])
        test_smtp_output('verify unknown', client.verify(email_part='test@mail.ru'), ['550', 'unknown', 'sender'])
        test_smtp_output('quit', client.quit(), ['221', 'closing'])
    print ''
\end{verbatim}

В результате запуска успешно-пройденные тесты отмечаются плюсами. Ошибки отмечаются минусами.

\section{Проверка на наличие утечек памяти}

Для проверки использовалась утилита Valgrind версии 3.13.0 с ручными изменениями в файле ./configure, необходимыми для успешной компиляции утилиты из исходников.

Результаты запуска команды \textit{valgrind --leak-check=yes ./smtp_server}:

\begin{verbatim}
...
\end{verbatim}

% Сборка программы описана в файле \textit{Makefile} системы сборки \textit{make}. Рис.~\ref{fig:make} нагенерили самодельные \textit{makesimple} и \textit{makefile2dot}, а также \textit{dot2tex} и \textit{dot}.

% \begin{figure} \centering \includegraphics[width=\textwidth]{include/Makefile_1_dot.pdf} \caption{Сборка программы} \label{fig:make} \end{figure}

% Отмечу, что за исключения целей типа \textit{all}, \textit{install}, \textit{clean}, \textit{tests}, все имена целей в файле систем сборки \textit{make} обычно совпадают с именами файлов (такой вот низкоуровневый инструмент). То есть вместо цели \textit{lexer} следует использовать цель \textit{src/lexer.c}.

\addcontentsline{toc}{chapter}{Выводы}
\chapter*{Выводы}

В результате выполнения курсового проекта достугнута поставленная цель, а именно разработано приложение smtp-сервера, обеспечивающего локальную доставку и добавление в очередь удаленной доставки.

В ходе выполнения работы были получены и закреплены следующие навыки:
\begin{itemize}
    \item проектирование и реализация сетевого протокола по имеющейся спецификации;
    \item реализация smtp-сервера на языке программирования Си;
    \item автоматизированное системное тестирование ПО сетевой службы;
    \item тестирования утечек памяти; 
    \item создание сценариев сборки ПО;
    \item использование latex и сценариев сборки для автогенерации расчетно-пояснительной записки.
\end{itemize}

Из-за ограниченного времени и ресурсов в данной работе не были реализованы следующие важные пункты:
\begin{itemize}
    \item проверка обратной зоны DNS, как базовый механизм от спам-рассылок (в принципе это не требовалось по условиям варианта);
    \item постановка сообщений в очередь почтового клиента для удаленной рассылки (из-за отсутствия команды);
    \item генерация документации утилитой doxygen.
\end{itemize}


\cleardoublepage
\phantomsection
\addcontentsline{toc}{chapter}{Приложение 1. Основные функции программы}
\chapter*{Приложение 1. Основные функции программы}

% Весь это раздел сгеренерировал doxygen из части комментированных исходников программы. В файле конфигурации \textbf{doxyggen.cfg} был отключён параметр \textbf{HAVE\_DOT}, поскольку для рисования графов вызовов используется \textit{cflow}.

% \input{include/files}

% \input{include/server-state_8h.tex}
% \input{include/server-state_8c.tex}
% \input{include/server-re_8h.tex}
% \input{include/server-cmd_8h.tex}
% \input{include/server-cmd_8c.tex}
% \input{include/server-parse_8h.tex}
% \input{include/server-parse_8c.tex}
% \input{include/server-run_8h.tex}
% \input{include/server-run_8c.tex}
% \input{include/server_8c.tex}


\cleardoublepage
\phantomsection
\addcontentsline{toc}{chapter}{Приложение 2. Графы вызова функций}
\chapter*{Приложение 2. Графы вызова функций}

% Поскольку функций много, графы вызовов разбиты на два рисунка. На рис.~\ref{fig:cflow01} показаны основные функции, на рис.~\ref{fig:cflow02}~-- функции обработки команд. Файл \textbf{cflow.ignore} содержит список функций (точнее, шабловнов поиска), использыемых программой \textit{grep} для удаления малоинтересных стандартных функций\footnote{Функции по работе с сокетами, ipc и привилегиями к малоинтересным ни в коем случае не относятся.}.

% \begin{figure} \centering \includegraphics[width=\textwidth]{include/cflow01_dot.pdf} \caption{Граф вызовов, основные функции} \label{fig:cflow01} \end{figure}

% \begin{figure} \centering \includegraphics[width=\textwidth]{include/cflow02_dot.pdf} \caption{Граф вызовов, функции обработки команд} \label{fig:cflow02} \end{figure}

% Графы созданы с помощью \textit{cflow}, \textit{cflow2dot}, \textit{dot}.

\end{document}
